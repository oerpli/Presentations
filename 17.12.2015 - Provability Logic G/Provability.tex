\documentclass{beamer}
\setbeamertemplate{navigation symbols}{}
\usetheme{Malmoe}
\usecolortheme{beaver}

%\beamertemplatenavigationsymbolsempty
\beamersetuncovermixins{\opaqueness<1>{25}}{\opaqueness<2->{15}}
\setbeamercovered{invisible}
%\usepackage{float}
\usepackage{amssymb}
\usepackage{wrapfig}
\usepackage{amsmath}
\usepackage[ngerman]{babel}
\usepackage[utf8]{inputenc}
\usepackage{float}
\usepackage{graphicx}
%\usepackage{wrapfig}
\usepackage{textcomp}
\usepackage{braket}
\usepackage{bbm}
\usepackage{framed}
\usepackage{bbold}
\usepackage{colortbl}
\usepackage{color}
\usepackage{ifthen}
%\usepackage{setspace}
\newcommand{\tikzfig}[2]{\begin{figure}[h]\begin{center}\input{./img/TikZ/#1.tex}\end{center}\end{figure}}
\newcommand{\tikzfigC}[2]{\begin{figure}[h]\begin{center}\input{./img/TikZ/#1.tex}\end{center}\caption{{#2}}\end{figure}}
\newcommand{\fig}[2]{\begin{figure}[h]\begin{center}\includegraphics[width = 0.5\textwidth]{./img/#1}\end{center}\caption{{#2}}\end{figure}}
\usepackage[T1]{fontenc}
\usepackage{amsthm}
\usepackage{bm}
\usepackage{amsbsy}
\usepackage{tikz}
\usepackage{xcolor}
	\usetikzlibrary{calc}
	\usetikzlibrary{arrows}
\usepackage{scalefnt}
\usetikzlibrary{shapes,shapes.multipart}
\usepackage{caption}
\addto\captionsngerman{
\renewcommand{\figurename}{Figure}%
\renewcommand{\tablename}{Tab.}%
}
\setlength{\parskip}{1.5ex plus0.5ex minus0.5ex}
\setlength{\parindent}{0em} 

\sloppy \frenchspacing \raggedbottom 





% gödel number macro
\newbox\gnBoxA
\newdimen\gnCornerHgt
\setbox\gnBoxA=\hbox{$\ulcorner$}
\global\gnCornerHgt=\ht\gnBoxA
\newdimen\gnArgHgt
\def\Godelnum #1{%
\setbox\gnBoxA=\hbox{$#1$}%
\gnArgHgt=\ht\gnBoxA%
\ifnum     \gnArgHgt<\gnCornerHgt \gnArgHgt=0pt%
\else \advance \gnArgHgt by -\gnCornerHgt%
\fi \raise\gnArgHgt\hbox{$\ulcorner$} \box\gnBoxA %
\raise\gnArgHgt\hbox{$\urcorner$}}
%%

% Prov makro
\DeclareMathOperator*{\Prov}{Prov}
\DeclareMathOperator*{\Var}{Var}

\DeclareMathOperator*{\Proov}{Proof}



\begin{document}
\part{ABS}
\title{The Provability Logic G}
\author{Abraham Hinteregger}
\institute{Vienna University of Technology}
\date{17.12.2015}
\titlepage
\setcounter{tocdepth}{4}
\AtBeginSection[]{
\begin{frame}
\frametitle{Chapter} 
\tableofcontents[currentsection,currentsubsection,hideothersubsections]
\end{frame}
}

\section{Incompleteness} 
\begin{frame}\frametitle{Incompleteness} 

\begin{block}{}
\begin{itemize}%[<+->]
\item 1931: Paper by Gödel ``Über formal unentscheidbare Sätze der Principia Mathematica und verwandter Systeme''\cite{goedel31}
\begin{itemize}
\item Incompleteness of Peano Arithmetic was shown by enumerating formulas of PA which enabled arguing about them in PA itself
\item Predicate $\Prov(x)$ -- formula $A$ s.t. $\Godelnum{A} = x$ is provable in PA. More formally $\Prov(x)  \equiv \exists p \Proov(p,x)$  where $\Proov(p,x)$ formalizes the notion that p is the GN of the formula that is a proof of the formula with GN x.
\item Constructed formula that asserts its own unprovability. 
\end{itemize}
%\item 1955: Löb's Theorem
\end{itemize}
\end{block}
\end{frame}


\subsection{Theorems}
\begin{frame}\frametitle{Incompleteness Theorems}
\begin{theorem}[First Incompleteness Theorem]
Every consistent and reasonably expressive arithmetic system contains sentences that are neither provable nor refutable.
\end{theorem}
\begin{theorem}[Second Incompleteness Theorem]
No consistent and moderately expressive arithmetic system can prove its own consistency.
\end{theorem}

\end{frame}
\subsection{Löb's Theorem}
\begin{frame}\frametitle{Löb's Theorem}
\begin{itemize}
\item<1-> Gödel proved his theorem by constructing a (true) formula that is asserts its own unprovability.
\item<2-> What about formula that asserts its own provability? \uncover<3->{Löb 1955
\begin{equation*}
\Box(\Box A \supset A) \supset \Box A
\end{equation*}}
\item<4-> Gödel's second incompleteness theorem is instance of Löb's theorem (it also implies it\cite{boolos}):
\begin{quote}
No \alert<5>{consistent system }can \alert<6>{prove its own consistency}
\end{quote}
\begin{align*}\alt<3-6>{
\uncover<5>{\lnot \Box \bot}
\uncover<9>{ \supset\lnot}
\uncover<6>{\alert<7>{\Box\lnot \Box \bot}}}
{\lnot \Box \bot\supset\lnot\Box\lnot \Box \bot}
\end{align*}
\begin{align*}
\uncover<9->{A \supset B \equiv \lnot B \supset \lnot A:&&\Box(\ \ \lnot \Box \bot\ \ )\supset\Box \bot}\\
\uncover<10->{\lnot A \equiv A \supset \bot:&&
\Box(\Box\bot \supset \bot)\supset\Box \bot}
\end{align*}

\end{itemize}

\end{frame}

\subsection{Fixed point theorem}

\begin{frame}
\frametitle{Fixed point theorem\cite{sep-prove}}
\begin{itemize}
\item<+->Variable $p$ only occurs in the scope of $\Box$ in a formula $A(p)$, e.g.:
\begin{equation*}
A(p) := \lnot \Box p
\end{equation*}
\item<+->Formula $B$ where the set of variables $\Var(B) = \Var(A)\smallsetminus\{p\}$ and 
\begin{equation*}
G \vdash B \equiv A(B)
\end{equation*}
is a fixed point of $A(p)$ and (almost) uniquely defined.
\item<+-> For $A(p)$ as defined above the fixed point $B$ is $\lnot \Box \bot$ and:
\begin{equation*}
G\vdash \lnot \Box\bot \equiv  \lnot\Box ( \lnot \Box\bot)
\end{equation*}
%\item<+-> Does someone recognize this formula?
\end{itemize}

\end{frame}


\section{The Logic G}
\subsection{Axioms}
\begin{frame}\frametitle{The Logic G -- Axioms}
\begin{itemize}
\item<1-> Distribution axiom
\begin{equation*}
\Box(A\supset B) \supset (\Box A \supset \Box B)
\end{equation*}
\item<1-> Necessitation
\begin{equation*}
\frac{A}{\Box A}
\end{equation*}
\item<2-> Löb's theorem
\begin{equation*}
\Box(\Box A \supset A) \supset \Box A 
\end{equation*}
\item<3-> Transitivity (consequence of Löb's theorem $\left[(B\land \Box B) / A \right]$)
\begin{equation*}
\Box B \supset \Box \Box B
\end{equation*}
\end{itemize}
\end{frame}

\subsection{Tableau for G}
\begin{frame}
\frametitle{Tableau rules}
\begin{itemize}
\item Start with \alt<0>{modal logic}{\alert<1>{K4} (transitive)} tableau \uncover<2->{and modify to \alert<2>{account for Löb's theorem}}
\begin{itemize}
\item If $\alpha$ is true in $\Gamma$ then $\alpha_1$ and $\alpha_2$ are true in $\Gamma$.
\item If $\beta$ is true in $\Gamma$ then $\beta_1$ or $\beta_2$ are true in $\Gamma$.
\item If $\pi$ is true in $\Gamma$ then $\pi_0$ is true in some $\Gamma^*$.
\item If $\nu$ is true in $\Gamma$ then $\nu_0$ is true in all $\Gamma^*$.
\item<1->\alert<1>{ If $\nu$ is true in $\Gamma$ then $\nu$ is true in all $\Gamma^*$.}
\item<2->\alert<2>{If $\pi$ is true in $\Gamma$ then $\bar \pi$ is true in some $\Gamma^*$}
\end{itemize}
\end{itemize}
\begin{tabular}{>{$}c<{$}|>{$}c<{$}||>{$}c<{$}|>{$}c<{$}|>{$}c<{$}}
\alpha&&\beta  \\\hline
 T A\land B & TA,TB&F A\land B & FA &FB\\\hline
 FA\lor B & FA ,FB &T A\lor B & FA & FB \\\hline
 FA \supset B &TA,FB&T A\supset B &FA &TB\\\hline
\end{tabular}
\begin{tabular}{>{$}c<{$}|>{$}c<{$}|>{$}c<{$}}
\hline
\nu&\!\!	F\Diamond A & FA,\alert<1>{\!  F\Diamond A}\\\hline
\nu&\!\!	T\Box A & TA,\alert<1>{\!\!T\Box A}\\\hline\hline
\pi&\!\!	T\Diamond A& TA\uncover<2>{,\alert<2>{\!F\Diamond A}}\\\hline
\pi&\!\!	F\Box A & FA\uncover<2>{,\alert<2>{\!\!T\Box A}}\\\hline
\end{tabular}\\
\end{frame}

\begin{frame}\frametitle{Examples}
On the blackboard:
\uncover<1->{
\begin{equation*}
\text{With K4 Tableau:  }\Box A
\end{equation*}
}
\uncover<3->{
\begin{equation*}
\text{With G Tableau: }\Box(\Box A \supset A)\supset \Box A
\end{equation*}
}

\begin{tabular}{>{$}c<{$}|>{$}c<{$}||>{$}c<{$}|>{$}c<{$}|>{$}c<{$}}
\alpha&&\beta  \\\hline
 T A\land B & TA,TB&F A\land B & FA &FB\\\hline
 FA\lor B & FA ,FB &T A\lor B & FA & FB \\\hline
 FA \supset B &TA,FB&T A\supset B &FA &TB\\\hline
\end{tabular}
\begin{tabular}{>{$}c<{$}|>{$}c<{$}|>{$}c<{$}}
\hline
\nu&\!\!	F\Diamond A & FA,{\!  F\Diamond A}\\\hline
\nu&\!\!	T\Box A & TA,{\!\!T\Box A}\\\hline\hline
\pi&\!\!	T\Diamond A& TA\uncover<2>{,{\!F\Diamond A}}\\\hline
\pi&\!\!	F\Box A & FA\uncover<2>{,{\!\!T\Box A}}\\\hline
\end{tabular}
\end{frame}

\subsection{Properties}
\begin{frame}
\frametitle{Properties\cite{fitting}}
\begin{itemize}
\item<+-> Model existence theorem: only holds for finite sets of signed formulas (finite set model existence theorem)
\item<+-> Tableau completeness for finite sets follows (strong tableau completeness theorem only for finite sets)
\item<+-> Local compactness does not hold for G
\begin{align*}
&\Diamond P_1\\
 \forall i:\qquad& \Box(P_i\supset \Diamond P_{i+1})
\end{align*}
\item<+-> Global compactness not known (at least neither Fittings nor Google knew)
\end{itemize}

\end{frame}



\section*{Stuff}
\subsection{References}

\bibliographystyle{IEEEtran} % kürzt automatisch vornamen ab und so zeug 
\begin{frame}\frametitle{References}
\bibliography{biblio}
\end{frame}

\subsection{Gödels incompleteness theorem II $\implies$ Löb's theorem}

\begin{frame}
\frametitle{Derivation of Löb's theorem}
Assume $T\nvDash A$ \\
$T\nvDash \Prov(\Godelnum{A})\supset A$\\
$T\cup \{\lnot A\} \nvDash\bot$\\
$T\cup \{\lnot A\} \nvDash \lnot \Prov_{T\cup \lnot A}(\Godelnum{\bot})$\\
$T\cup \{\lnot A\} \nvDash \lnot \Prov_{T}(\Godelnum{\lnot A \supset \bot})$\\
$T\cup \{\lnot A\} \nvDash \lnot \Prov_{T}(\Godelnum{A})$\\
$T \nvDash \lnot A \supset \lnot \Prov_{T}(\Godelnum{A})$\\
$T \nvDash \Prov_{T}(\Godelnum{A}) \supset A$\\






\end{frame}
\end{document}
