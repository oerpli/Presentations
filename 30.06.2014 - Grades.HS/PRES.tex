\documentclass{beamer}
\setbeamertemplate{navigation symbols}{}
\usetheme{Malmoe}
\usecolortheme{beaver}

%\beamertemplatenavigationsymbolsempty
\beamersetuncovermixins{\opaqueness<1>{25}}{\opaqueness<2->{15}}

%\usepackage{float}
\usepackage{amssymb}
\usepackage{wrapfig}
\usepackage{amsmath}
\usepackage[ngerman]{babel}
\usepackage[utf8]{inputenc}
\usepackage{float}
\usepackage{graphicx}
%\usepackage{wrapfig}
\usepackage{textcomp}
\usepackage{braket}
\usepackage{bbm}
\usepackage{framed}
\usepackage{bbold}
\usepackage{colortbl}
\usepackage{color}
\usepackage{ifthen}
%\usepackage{setspace}
\newcommand{\tikzfig}[2]{\begin{figure}[h]\begin{center}\input{./img/TikZ/#1.tex}\end{center}\end{figure}}
\newcommand{\tikzfigC}[2]{\begin{figure}[h]\begin{center}\input{./img/TikZ/#1.tex}\end{center}\caption{{#2}}\end{figure}}
\newcommand{\fig}[2]{\begin{figure}[h]\begin{center}\includegraphics[width = 0.5\textwidth]{./img/#1}\end{center}\caption{{#2}}\end{figure}}
\usepackage[T1]{fontenc}
\usepackage{amsthm}
\usepackage{bm}
\usepackage{amsbsy}
\usepackage{tikz,pgfplots}
\usepackage{xcolor}
\usepackage{scalefnt}
\usepackage{caption}
\usepackage{listings}


\definecolor{gray_ulisses}{gray}{0.55}
\definecolor{castanho_ulisses}{rgb}{0.71,0.33,0.14}
\definecolor{preto_ulisses}{rgb}{0.41,0.20,0.04}
\definecolor{green_ulises}{rgb}{0.2,0.75,0}

\lstdefinelanguage{HaskellUlisses} {
	basicstyle=\ttfamily\footnotesize,
	sensitive=true,
	morecomment=[l][\color{gray_ulisses}\ttfamily\scriptsize]{--},
	morecomment=[s][\color{gray_ulisses}\ttfamily\scriptsize]{\{-}{-\}},
	morestring=[b]",
	stringstyle=\color{red},
	showstringspaces=false,
%	numberstyle=\tiny,
	numberblanklines=true,
	showspaces=false,
	breaklines=true,
	showtabs=false,
	emph=
	{[1]
		FilePath,IOError,abs,acos,acosh,all,and,any,appendFile,approxRational,asTypeOf,asin,
		asinh,atan,atan2,atanh,basicIORun,break,catch,ceiling,chr,compare,concat,concatMap,
		const,cos,cosh,curry,cycle,decodeFloat,denominator,digitToInt,div,divMod,drop,
		dropWhile,either,elem,encodeFloat,enumFrom,enumFromThen,enumFromThenTo,enumFromTo,
		error,even,exp,exponent,fail,filter,flip,floatDigits,floatRadix,floatRange,floor,
		fmap,foldl,foldl1,foldr,foldr1,fromDouble,fromEnum,fromInt,fromInteger,fromIntegral,
		fromRational,fst,gcd,getChar,getContents,getLine,head,id,inRange,index,init,intToDigit,
		interact,ioError,isAlpha,isAlphaNum,isAscii,isControl,isDenormalized,isDigit,isHexDigit,
		isIEEE,isInfinite,isLower,isNaN,isNegativeZero,isOctDigit,isPrint,isSpace,isUpper,iterate,
		last,lcm,length,lex,lexDigits,lexLitChar,lines,log,logBase,lookup,map,mapM,mapM_,max,
		maxBound,maximum,maybe,min,minBound,minimum,mod,negate,not,notElem,null,numerator,odd,
		or,ord,otherwise,pi,pred,primExitWith,print,product,properFraction,putChar,putStr,putStrLn,quot,
		quotRem,range,rangeSize,read,readDec,readFile,readFloat,readHex,readIO,readInt,readList,readLitChar,
		readLn,readOct,readParen,readSigned,reads,readsPrec,realToFrac,recip,rem,repeat,replicate,return,
		reverse,round,scaleFloat,scanl,scanl1,scanr,scanr1,seq,sequence,sequence_,show,showChar,showInt,
		showList,showLitChar,showParen,showSigned,showString,shows,showsPrec,significand,signum,sin,
		sinh,snd,span,splitAt,sqrt,subtract,succ,sum,tail,take,takeWhile,tan,tanh,threadToIOResult,toEnum,
		toInt,toInteger,toLower,toRational,toUpper,truncate,uncurry,undefined,unlines,until,unwords,unzip,
		unzip3,userError,words,writeFile,zip,zip3,zipWith,zipWith3,listArray,doParse
	},
	emphstyle={[1]\color{blue}},
	emph=
	{[2]
		Bool,Char,Double,Either,Float,IO,Integer,Int,Maybe,Ordering,Rational,Ratio,ReadS,ShowS,String,
		Word8,InPacket
	},
	emphstyle={[2]\color{castanho_ulisses}},
	emph=
	{[3]
		case,class,data,deriving,do,else,if,import,in,infixl,infixr,instance,let,
		module,of,primitive,then,type,where
	},
	emphstyle={[3]\color{preto_ulisses}\textbf},
	emph=
	{[4]
		quot,rem,div,mod,elem,notElem,seq
	},
	emphstyle={[4]\color{castanho_ulisses}\textbf},
	emph=
	{[5]
		EQ,False,GT,Just,LT,Left,Nothing,Right,True,Show,Eq,Ord,Num
	},
	emphstyle={[5]\color{preto_ulisses}\textbf}
}

%\lstnewenvironment{code}
%{\textbf{Haskell Code} \hspace{1cm} \hrulefill \lstset{language=HaskellUlisses}}
%{\hrulesmallskip}

\usetikzlibrary{calc,arrows,external,shapes,shapes.multipart}
%\tikzexternalize[prefix=figures/]


\addto\captionsngerman{
\renewcommand{\figurename}{Figure}%
\renewcommand{\tablename}{Tab.}%
}
\setlength{\parskip}{1.5ex plus0.5ex minus0.5ex}
\setlength{\parindent}{0em} 

\sloppy \frenchspacing \raggedbottom 


\begin{document}
\part{ABS}
\title{Grades.HS}
\author{Abraham Hinteregger, BSc}
\institute{Vienna University of Technology}
\date{30.06.2014}
\titlepage
\setcounter{tocdepth}{4}




\section{Grades.HS} 
\subsection{Gewünschte Features}
\begin{frame}[fragile]
\frametitle{Gewünschte Features} 
\begin{itemize}
\item Notenstatistik
\item ohne Excel
\item verschiedene Notensysteme (at,ch,us, \ldots)
\end{itemize}
\end{frame}

\subsection{Umsetzung}
\begin{frame}[fragile]\frametitle{Umsetzung}

\begin{itemize}
\item Typ für Fach und Note mit verschiedenen Untertypen
\item Zugriff mittels lens Package
\item State entspricht einer Liste aus Fächern
\item Typ für verschiedene Manipulationen von State
\item Typ für verschiedene Arten der Schnittberechnung
\end{itemize}
\end{frame}


\begin{frame}[fragile]
\frametitle{applyAction }
\begin{lstlisting}[language=HaskellUlisses]
-- | Applies an action to a list of subjects and returns the modified list.
applyAction :: LAction -> [LSubject]-> [LSubject]
applyAction (RemRes i) s= s & ix i . result .~ Nothing
applyAction (AddRes i r) s= s & ix i . result .~Just r
applyAction (AddSub n) s= s ++ [n]
applyAction _ s= s
\end{lstlisting}
\end{frame}



\begin{frame}[fragile]\frametitle{GUI \& State}

\begin{itemize}
\item GUI mit Threepenny-GUI
\item State mit IORef
\begin{itemize}
\item Durchreichen der Referenz und des Fensters 
\item Bei Änderungen IOModifyRef
\item Manipulation des Fensters (nicht inkrementell sondern immer vollständig
\end{itemize}
\end{itemize}
\end{frame}



\begin{frame}[fragile]
\frametitle{main :: IO()}
\begin{lstlisting}[language=HaskellUlisses]
main :: IO()
main = do
	startGUI config (setup state)

setup :: [Maybe LSubject] -> Window -> UI ()
setup s w = void $ do
    io   <- liftIO $ newIORef (catMaybes s)
    view <- mkView (w,io) s
    getBody w UI.# set UI.children [view]
\end{lstlisting}
\end{frame}

\section{Erfahrung}
\subsection{Positiv \& Negativ}
\begin{frame}[fragile]
Positiv
\begin{itemize}
\item Keine Probleme mit verschiedenen Package Versionen o.Ä.
\item Debuggen mit :t und :info ist recht angenehm
\item Recht einfach nachzuvollziehen wo etwas schiefgeht.
\end{itemize}
Negativ
\begin{itemize}
\item Dokumentation von Threepenny-GUI und FRP etwas dürftig
\item Instanzierung von \lstinline[language=HaskellUlisses]{Read a} extrem mühsam
\item An Fehlerbehandlung gescheitert
\end{itemize}
\end{frame}

\subsection{Sonstiges}
\begin{frame}Sonstiges
\begin{itemize}
\item GUI- Code bläht sich recht schnell auf (möglicherweise meinem Stil geschuldet)
\item gigantische Binaries
\end{itemize}
\end{frame}

\end{document}