\documentclass{beamer}
\setbeamertemplate{navigation symbols}{}
\usetheme{Malmoe}
\usecolortheme{beaver}
%\usepackage{natbib}   
%\bibliographystyle{plainnat}
\bibliographystyle{apalike}   % Or any other style you like
%\bibliographystyle{natbib} % kürzt automatisch vornamen ab und so zeug 

%\beamertemplatenavigationsymbolsempty
\beamersetuncovermixins{}{}
%\setbeamercovered{invisible}
%\usepackage{float}
\usepackage{amssymb}
\usepackage{wrapfig}
\usepackage{amsmath}
\usepackage[ngerman]{babel}
\usepackage[utf8]{inputenc}
\usepackage{float}
\usepackage{graphicx}
%\usepackage{wrapfig}
\usepackage{textcomp}
\usepackage{braket}
\usepackage{bbm}
\usepackage{framed}
\usepackage{ifthen}
\usepackage{bbold}
\usepackage{colortbl}
\usepackage{xifthen}
\usepackage{color}
\usepackage{ifthen}
%\usepackage{setspace}
\newcommand{\tikzfig}[2]{\begin{figure}[h]\begin{center}\input{./img/TikZ/#1.tex}\end{center}\end{figure}}
\newcommand{\tikzfigC}[2]{\begin{figure}[h]\begin{center}\input{./img/TikZ/#1.tex}\end{center}\caption{{#2}}\end{figure}}
\newcommand{\fig}[2]{\begin{figure}[h]\begin{center}\includegraphics[width = 0.5\textwidth]{./img/#1}\end{center}\caption{{#2}}\end{figure}}
\usepackage[T1]{fontenc}
\usepackage{amsthm}
\usepackage{bm}
\usepackage{amsbsy}
\usepackage{tikz}
\usepackage{xcolor}
\usepackage{scalefnt}
\usetikzlibrary{shapes,shapes.multipart}
\usepackage{caption}
\addto\captionsngerman{
\renewcommand{\figurename}{Figure}%
\renewcommand{\tablename}{Tab.}%
}
\setlength{\parskip}{1.5ex plus0.5ex minus0.5ex}
\setlength{\parindent}{0em} 

\sloppy \frenchspacing \raggedbottom 


\usetikzlibrary{shapes,snakes,calc,arrows}
%\usetikzlibrary{external}
%\tikzexternalize[prefix=external]
\usepackage{verbatim}


% gödel number macro
\newbox\gnBoxA
\newdimen\gnCornerHgt
\setbox\gnBoxA=\hbox{$\ulcorner$}
\global\gnCornerHgt=\ht\gnBoxA
\newdimen\gnArgHgt
\def\Godelnum #1{%
\setbox\gnBoxA=\hbox{$#1$}%
\gnArgHgt=\ht\gnBoxA%
\ifnum     \gnArgHgt<\gnCornerHgt \gnArgHgt=0pt%
\else \advance \gnArgHgt by -\gnCornerHgt%
\fi \raise\gnArgHgt\hbox{$\ulcorner$} \box\gnBoxA %
\raise\gnArgHgt\hbox{$\urcorner$}}
%%

% Prov makro
\DeclareMathOperator*{\Prov}{Prov}
\DeclareMathOperator*{\Var}{Var}

\DeclareMathOperator*{\Proov}{Proof}

\newcommand{\NP}{\ensuremath{\mathcal{NP}}}
\begin{document}
\part{ABS}
\title{Geometric Independet Set Problem}
\author{Abraham Hinteregger}
\institute{Vienna University of Technology}
\date{13.4.2016}
\titlepage
\setcounter{tocdepth}{3}
\AtBeginSection[]{
\begin{frame}
\frametitle{Chapter} 
\tableofcontents[currentsection,currentsubsection,hideothersubsections]
\end{frame}
}


\section{Intersection graph}

\subsection{Introduction} 
\begin{frame}\frametitle{Intersection graph} 

 \begin{columns}[T]
    \begin{column}{.5\textwidth}
  %  \begin{block}
  {
     \begin{itemize}[<+->]
\item Given an arrangement $A$ of geometric objects
\item The intersection graph $G$ has a vertice $v_i$ for every object $O_i \in A$
\item If two objects intersect there is an edge between them in $G$
\end{itemize}
}
    %\end{block}
    \end{column}
    \begin{column}{.5\textwidth}
%    \begin{block}
    {
    \begin{tikzpicture}[scale=0.5]
    \newcommand{\centers}{(0,1.3),(0,0),(0.7,1.5),(2,1),(5,3),(4,3),(0,5),(1.6,3.5),(6,0.5)}%

    \uncover<1-3>{ \foreach [count=\i] \coord in \centers{\draw \coord circle(1) node{};}}    
    \uncover<4-4>{ \foreach [count=\i] \coord in \centers{\draw[opacity=0.25] \coord circle(1) node{};}}    
    \uncover<2-4>{ \foreach \coord in \centers{\fill \coord circle(0.15);}}        
    
\uncover<3-4>{
%     \foreach [count=\i]\a in \centers{
%      	\foreach [count=\j]\b in \centers{
%      	\ifthenelse{\i < \j}{      	}{};
%      }}
         \draw[thick] (0,1.3) -- (0,0) --(0.7,1.5) --(2,1) -- cycle;
    \draw[thick] (0,1.3) -- (0.7,1.5);
    \draw[thick] (5,3) --(4,3);
}


    \end{tikzpicture}
    }
 %   \end{block}
    \end{column}
  \end{columns}
\end{frame}

\begin{frame}
\frametitle{Intersection graph}
\begin{itemize}
\item Can be defined for arbitrary geometric objects (example on last slide was a unit disk graph - UDG)
\item Given the arrangement of objects can be calculated in polynomial time $O(n^2)$ if complexity of objects is bounded
\item<2-> Some subsets of IG and possible applications
\begin{itemize}
\item<2-> Clique ($K_n$ subgraphs of G) of interval graph (IG for intervals $ i \in \mathbb R^2$) can be used for scheduling (similar to doodle)
\item<3-> Path between two vertices on IG of geometric shapes corresponds to a path between two points without leaving perimeter of shapes (e.g. cheapest way from A to B with fee on borders)
\item<4-> Independent set (non-adjacent vertices) of IG of geometric shapes corresponds to non-intersecting subset of the shapes in the arrangement
\end{itemize}
\end{itemize}
\end{frame}




\subsection{Independent set of IG}

\begin{frame}
\frametitle{Map labelling}
 \begin{columns}[T]
    \begin{column}{.5\textwidth}
   %  \begin{block}{
\begin{itemize}
\item<1-> Given a 2D map with various points of interest -- which labels should be drawn?
\item<3-> Labels must not intersect each other 
\item<4-> Variants of the problem:
\begin{itemize}
\item Labels with size- constraints (e.g.: uniform height)
\item Labels are allowed in a radius around corresponding datapoint
\end{itemize}
\end{itemize}
%}
%    \end{block}
    \end{column}
    \begin{column}{.5\textwidth}
%    \begin{block}
    {
    \begin{tikzpicture}[scale=0.5]
   \newcommand{\centers}{(0,0),(5,3),(2,1),(0,5),(1.6,3.5),(6,0.5), (2,8),(5,8),(4,6),(1,10),(0.5,1),(0,1),(4,3)}
   \uncover<1->{ \foreach \coord in \centers{\fill \coord circle(0.15);}}        
   \uncover<2-2>{ \foreach [count=\i] \coord in \centers{\ifthenelse{\i < 111}{\draw \coord node[below]{City \i};}}}{};  
   \uncover<3->{ \foreach [count=\i] \coord in \centers{\ifthenelse{\i < 11}{\draw \coord node[below]{City \i};}}}{};  
    \end{tikzpicture}
    }
 %   \end{block}
    \end{column}
  \end{columns}
\end{frame}


\begin{frame}
\frametitle{Construction of wireless network \cite{chamaret} }
 \begin{columns}[T]
    \begin{column}{.63\textwidth}
%  \begin{block}
  {
\begin{itemize}
\item<1-> Maximize area with reception
\item<2-> Building base stations may be expensive $\rightarrow$ overlap is inefficient.
\item<5-> Can also be used for assigning frequencies
\begin{itemize}
\item More or less a graph coloring problem
\item Smallest number of necessary frequencies ($\hat{=}$ colors) is the chromatic number \cite{mcdiarmid} which is related to IS.
\end{itemize}
\end{itemize}
}

    \end{column}
    \begin{column}{.37\textwidth}

    {
    \begin{tikzpicture}[scale=0.5]
   \newcommand{\centers}{(0,0),(2,1),(1.6,3.5),(0,5),(5,3),(0.6,7),(4,6)}

   \uncover<1-2>{ \foreach \coord in \centers{\draw \coord circle(2);}}        
%   \uncover<2-2>{ \foreach [count=\i] \coord in \centers{\ifthenelse{\i < 111}{\draw \coord node[below]{City \i};}}}{};  

   \uncover<2->{\draw (0,0) -- (2,1) -- (1.6,3.5) -- (0,5)-- (0.6,7);}
\uncover<3->{ \foreach \coord in \centers{\draw[gray!50] \coord circle(2);}}        
\newcommand{\centerss}{(0,0),(1.6,3.5),(5,3),(0.6,7),(4,6)}
\uncover<3->{ \foreach \coord in \centerss{\draw[ultra thick, darkred]\coord circle(0.25);}}        
\uncover<3->{ \foreach \coord in \centerss{\draw \coord circle(2);}}        
\uncover<1->{ \foreach \coord in \centers{\fill \coord circle(0.15);}}        

    \end{tikzpicture}
    }

    \end{column}
  \end{columns}
\end{frame}


\subsection{PTAS}

\begin{frame}
\frametitle{Polynomial time approximation scheme}
\begin{itemize}
\item<1-> Finding IS of graph \NP - hard --- therefore finding approximative solutions in polynomial time would already be nice.
\item<1-> Some random estimate rather uninteresting---approximation should have some quality estimate.
\uncover<1->{
\begin{alertblock}{Polynomial time $\rho$ -- approximation scheme}
Finds solution with a solution quality that is at least $\frac{S_{OPT}}{\rho}$ in polynomial time (for fixed $\rho> 1 $)
\end{alertblock}}
\end{itemize}
\end{frame}


% present one or both - if only one add to background or to Algorithms (rename to whatever)
\section{M(W)IS in Unit Disk Graphs}
\subsection{Problem description}
\begin{frame}
\frametitle{PTAS for M(W)IS problem in UDG \cite{nieberg}}
\begin{itemize}
\item Geometric objects are only disks with radius 1
\item Two variants of the problem -- with or without geometric representation
\item If geometric representation is known seperation alongside some grid is possible which allows more efficient approaches (shifting strategy \cite{fonseca})
\item Finding geometric representation from intersection graph is \NP- hard \cite{nphard}
\end{itemize}
\end{frame}
\subsection{Unweighted MIS}

\begin{frame}
\frametitle{Unweighted MIS of UDG}

Given: Some graph $G = (E,V)$ which is a UDG iff there is a mapping $f: V \rightarrow \mathbb R^2$ s.t.:
\begin{align*}(u,v) \in E \leftrightarrow ||f(u) - f(v) || \leq 2 \qquad \forall u,v \in V, (u\neq v)\end{align*}
%\item
Desired: Subset $I \subset V$ s.t. $|I|\cdot \rho \geq \alpha(G)$\\ $\alpha(G)$ \ldots size maximum IS \\[1em]
\uncover<2->{
\begin{itemize}
\item Idea of algorithm
\begin{enumerate}
\item Take subsets of graph with bounded size
\item Calculate MIS of subsets
\item Combine MIS of subsets to get IS of whole graph
\end{enumerate}
\end{itemize}}
\end{frame}

\newcommand{\vv}{v_0}
\begin{frame}
\frametitle{Step 1: Subsets of finite size}

Define the following sets for some arbitary node $\vv\in V$ 
\begin{equation*}
N_r := \{w| w\in V, w \text{ has distance at most $r$ from $\vv$}\}
\end{equation*}
\uncover<2->{and calculate MIS $I_r \subset N_r$ for $r= 0,1,\ldots \bar r$ where $\bar r$ is defined as the smallest $r$ s.t. 
 \begin{equation*}
 I_{r+1} > \rho|I_r|
\end{equation*}
does not hold.}
%\end{itemize}
\end{frame}

\begin{frame}
\frametitle{Bounding size of $I_{\bar r}$}
\begin{equation*}
\forall w\in N_r: \quad ||f(\vv) - f(w)|| \leq 2r
\end{equation*}
therefore it's possible to draw a circle with radius $R = 2r +1$ and centerpoint $\vv$ that contains all disks in $I_r$ and
\begin{align*}
|I_r| \leq \pi R^2 / \pi = O(r^2).
\end{align*}
By definition of $\bar r$ 
\begin{equation*}
|I_r| > \rho |I_{r-1}| > \ldots > \rho^r |I_0| = \rho^r
\end{equation*}also holds. Combining these two results:
\begin{equation*}
\rho^r < |I_r| \leq O(r^2)
\end{equation*} $\implies$ constant (depending only on $\rho$) bound on $\bar r$
\end{frame}

\begin{frame}
    \begin{tikzpicture}[scale=0.5]
    \
    \coordinate (V) at (2,2);
    \coordinate (V2) at (3.6,2.2);
   \newcommand{\None}{(2,0.6),(0,2.1),(V2)}
   \newcommand{\Ntwo}{(5,1.5),(4,3.8)}

   \uncover<1->{  
   	\fill (V) circle(0.15);
	\foreach \coord in \None{\fill \coord circle(0.15);}
	\foreach \coord in \Ntwo{\fill \coord circle(0.15);}
	\foreach \coord in \None{\draw (V) --\coord;}
	\foreach \coord in \Ntwo{\draw (V2) --\coord;}
	
}        
   \uncover<2->{
	 \draw (V) circle(1);
   	 \foreach \coord in \None{\draw \coord circle(1);}
	\foreach \coord in \Ntwo{\draw \coord circle(1);}
}        
\uncover<3->{\draw[] ($(V)+(-0.10,0.2)$)--+(-0.6,1)node[above]{$\vv$};}       
\coordinate(CAP) at (3,1);
\node<1> [below=1cm, align=flush center,text width=5cm] at (CAP){UDG};
\node<2-> [below=1cm, align=flush center,text width=5cm] at (CAP){UD arrangement \& UDG};
\draw[white] (-1.4,5) rectangle(7.2,-2.5);\end{tikzpicture}
\begin{tikzpicture}[scale=0.5]
    \coordinate (V) at (2,2);
    \coordinate (V2) at (3.6,2.2);
   \newcommand{\None}{(2,0.6),(0,2.1),(V2)}
   \newcommand{\Ntwo}{(5,1.5),(4,3.8)}

      
   \uncover<3->{

	\foreach \coord in \None{\fill[gray!50] \coord circle(0.15);}
	\foreach \coord in \Ntwo{\fill[gray!50] \coord circle(0.15);}
	\foreach \coord in \None{\draw[gray!50] (V) --\coord;}
	\foreach \coord in \Ntwo{\draw[gray!50] (V2) --\coord;}
   	 \foreach \coord in \None{\draw[gray!50] \coord circle(1);}
	\foreach \coord in \Ntwo{\draw[gray!50] \coord circle(1);}
     	\fill[] (V) circle(0.15);
     	\draw[thick] (V) circle(1);	
}   

   \uncover<4->{  


\draw[ultra thick,darkred] (V) circle(0.25);
}  

\coordinate(CAP) at (3,1);
 \node<3> [below=1cm, align=flush center,text width=5cm] at (CAP){$N_0$};
 \node<4-> [below=1cm, align=flush center,text width=5cm] at (CAP){$N_0$ (black) and $I_0$ (red)};
\draw[white] (-1.4,5) rectangle(7.2,-2.5);\end{tikzpicture}
\\
    \begin{tikzpicture}[scale=0.5]
    \
    \coordinate (V) at (2,2);
    \coordinate (V2) at (3.6,2.2);
   \newcommand{\None}{(2,0.6),(0,2.1),(V2)}

   \newcommand{\Ntwo}{(5,1.5),(4,3.8)}
  
   \uncover<5->{
	\draw<5>[thick] (V) circle(1);
   	\draw<6->[thick,gray!50] (V) circle(1);       
	\foreach \coord in \Ntwo{\draw[gray] \coord circle(1);}
	\foreach \coord in \None{\draw[thick] \coord circle(1);}
	\foreach \coord in \Ntwo{\draw[gray] (V2) --\coord;}
	\foreach \coord in \None{\draw (V) --\coord;}
   	\fill (V) circle(0.15);
}    
\foreach \coord in \Ntwo{\draw<6->[thick] (V2) --\coord;}
\foreach \coord in \None{\draw<6->[thick] (V) --\coord;}
\foreach \coord in \None{\draw<6->[ultra thick,darkred] \coord circle(0.25);}
\foreach \coord in \None{\fill<5->[] \coord circle(0.15);}
\foreach \coord in \Ntwo{\fill<5->[gray!50] \coord circle(0.15);}

\coordinate(CAP) at (3,1);
\node<5> [below=1cm, align=flush center,text width=5cm] at (CAP){$N_1$};
\node<6-> [below=1cm, align=flush center,text width=5cm] at (CAP){$N_1$ and $I_1$}; 
\draw[white] (-1.4,5) rectangle(7.2,-2.5);\end{tikzpicture}
\begin{tikzpicture}[scale=0.5] % i guess this is the bist picture of them all. use this for copypasting later
\coordinate (V) at (2,2);
\coordinate (V2) at (3.6,2.2);
\newcommand{\None}{(2,0.6),(0,2.1),(V2)}
\newcommand{\Nonee}{(2,0.6),(0,2.1)}
\newcommand{\Ntwo}{(5,1.5),(4,3.8)}

\draw<7>[thick] (V) circle(1);
\draw<7>[thick] (V2) circle(1);
\foreach \coord in \Ntwo{\fill<7-> \coord circle(0.15);}
\foreach \coord in \None{\fill<7-> \coord circle(0.15);}
\draw<8->[thick,gray!50] (V) circle(1);
\draw<8->[thick,gray!50] (V2) circle(1);
\foreach \coord in \Nonee{\draw<7->[thick] \coord circle(1);}
\foreach \coord in \Ntwo{\draw<7->[thick] \coord circle(1);}
\foreach \coord in \Ntwo{\draw<7->[thick] (V2) --\coord;}
\foreach \coord in \None{\draw<7->[thick] (V) --\coord;}
\foreach \coord in \Nonee{\draw<8->[ultra thick,darkred] \coord circle(0.25);}
\foreach \coord in \Ntwo{\draw<8->[ultra thick,darkred] \coord circle(0.25);}
\fill<7> (V) circle(0.15);
\fill<7> (V2) circle(0.15);
\fill<8->[gray] (V) circle(0.15);
\fill<8->[gray] (V2) circle(0.15);
\coordinate(CAP) at (3,1);
\node<7> [below=1cm, align=flush center,text width=5cm] at (CAP){$N_2$};
\node<8> [below=1cm, align=flush center,text width=5cm] at (CAP){$N_2$ and $I_2$};
\node<9-> [below=1cm, align=flush center,text width=5cm] at (CAP){$N_2$ and $I_2$, $\bar r = 1$};
\draw[white] (-1.4,5) rectangle(7.2,-2.5);
\end{tikzpicture}
\end{frame}



\begin{frame}
From the definition of $N_r$ it follows that for the subgraph $G' = G[N_{\bar r+1}]$ the maximum independet set size is bounded:
\begin{equation*}
\alpha(G') \leq \rho|I_{\bar r}|.
\end{equation*}
$H = G\setminus G'$ has no vertices adjacent to vertices in $N_{\bar r}$ and therefore a $\rho$-approximate IS for $H$ ($I_H$) combined with $I_{\bar r}$ yields a $\rho$-approximate IS for G.
\begin{align*}
\alpha(G) \leq \alpha(H) + \alpha(G') \leq \rho|I_H \cup I_{\bar r}|
\end{align*}
\end{frame}

\begin{frame}
\frametitle{Algorithm}
\begin{enumerate}
\item Define sets $N_r$ for arbitary node $v \in V$ and calculate independent sets $I_r$ until stopping criterion reached
\begin{itemize}
\item Remember that $r$ has constant bound $\implies$ determining $I_r$ is possible in $O(n^{C^2})$.
\end{itemize}
\item Repeat step 1. for subgraph $H= G \setminus N_{\bar r +1}$ until $H = \varnothing$.
\item Output independent set $I = \bigcup I_{\bar r}$ where $I_{\bar r}$ are alle the IS found in step 1. 
\end{enumerate}
\end{frame}

\subsection{Weighted MIS}
\begin{frame}
\frametitle{Algorithm for maximum weighted independet set problem}
%\begin{itemize}
When defining sets $N_r$ start with $v$ such that 
\begin{equation*}
\omega(v) \geq \omega(v') \forall v' \in V \qquad (=\text{argmax } \omega(v)) 
\end{equation*}and use
\begin{equation*}
\omega(I_{r+1}) > \rho \omega(I_r)
\end{equation*}as stopping criterion (for $\bar r$) to get bound on the weight of the IS of the subsets.
%\end{itemize}
\end{frame}

\section{MIS for Rectangles}
\subsection{MIS for arbitrary rectangles}
\begin{frame}
\frametitle{MIS for arbitrary rectangles \cite{agarwallabel}}
\begin{itemize}
\item Geometric objects are rectangles in $\mathbb R^2$ with parallel edges
\item Desired: Biggest subset of those rectangles that don't intersect each other
\item $\log n$ -- approximate algorithm for this case with $O(n\log n)$ runtime: 
\begin{itemize}
\item Divide and concquer: Split into subsets and calculate independet sets for them
\item Combine IS of subsets.
\end{itemize}
\end{itemize}
\end{frame}

\begin{frame}
\frametitle{Maximum independet set for arbitrary rectangles}
Given: Set $S$ of rectangles.
\begin{itemize}
\item<1-> Sort edges by $x$ and $y$ coordinates --- $O(n\log n)$
\item<2-> \begin{enumerate}
\item<2-> Find median in $x$ coordinate --- $O(n)$
\item<3-> Divide rectangles in 3 sets: $S_M$ (intersecting median), $S_L$ and $S_R$ (left and right of median)
\item<4-> Apply algorithm 1-4 recursively to $S_L$ and $S_R$ if their size is $\geq 2$ to get approximate solution, else calculate MIS in $O(1)$.
\item<5-> Calculate MIS of $S_M$ with greedy approach (outlined on blackboard) --- $O(n)$
\item<6-> Take either $I_L \cup I_R$ or $I_M$ 
\end{enumerate}
\end{itemize}

\uncover<7->{Overall runtime: $O(n\log n)$}
\end{frame}
%\item TODO: create slides for algorithm in \cite{agarwallabel}.
%\item Algorithm is more or less: sort edges of rectangles, find median, subdivide in rectangles on left and right side of meidan and combine their IS with those of  rectangles intersectiong median.
%\item Uses the known height of rectangles combined with their y coordinate to reduce problem difficulty by seperating into set of easier problems ($\rightarrow$ dynamic programming)


\subsection{MIS for unit height rectangles}
\begin{frame}
\frametitle{MIS for unit height rectangles \cite{agarwallabel}}
\begin{itemize}[<+->]
\item Map labels often have similar/identical height
%\item Could be exploited for better results
\item Use height to split problem in small subproblems that don't depend on each other
\item Apply greedy scheme from before for subproblems
\end{itemize}
\end{frame}

\begin{frame}
\frametitle{MIS for unit height rectangles (UHR)}
\uncover<1->{Draw $m+1$ horizontal lines that fulfill the following criteria:
\begin{itemize}
\item $y$ - distance $>1$
\item Each line intersects $\geq 1$ rectangles
\item Each rectangle is intersected by $1$ line.
\end{itemize}
}
\uncover<2->{Solve MIS for subsets $S_0$ to $S_m$ (one for every line) with greedy algorithm.\\[1em]}
\uncover<3->{Take union of either odd or even subsets}
\end{frame}

%\begin{frame}
%\frametitle{Possible improvements:}
%\end{frame}

\subsection{Take away points}
\begin{frame}
\frametitle{Take away points}
For difficult problems use all available information about objects at hand to make problem solvable (or at least get good approximate solutions):
\begin{itemize}
\item 1st algorithm (UDG): minimum area of a independet set of geometric shapes used to get an upper bound on size of subsets 
\item 2nd algorithm (greedy heuristic): reduce dimensionality of problem by line that intersects all rectangles
\item 3rd algorithm (UHR): decompose problem into subproblems that can be solved easily
\end{itemize}

\end{frame}




%\section*{Stuff}
%\subsection{References}
\begin{frame}[allowframebreaks]\frametitle{References}
\bibliography{alg}
\end{frame}

\begin{frame}
\frametitle{Remarks I}
Robustness of PTAS for M(W)IS on UDG
\begin{itemize}
\item<2-> Polynomial runtime only possible on UDG \uncover<3->{as it assumes bounded area of IS with given size}
\item<3-> UDG are subset of all graphs --- proving that graph is UDG is \NP- hard
\item<4-> Robustness would mean: Determine in polynomial runtime whether runtime will be polynomial. \uncover<5->{Is this possible?}
\item<5-> Actually possible in this case:  If $|I_r > (2r+1)^2$ is found $\implies$ graph is not UDG.
\end{itemize}
\end{frame}

\begin{frame}
\frametitle{Remarks II}
Improving PTAS for unit height rectangles to {$(1+\varepsilon)$-- approximative} scheme:
\begin{itemize}[<+->]
\item Algorithm described solves $n$ independet subproblems optimally and discards results of $\frac{n}{2}$
\item Discard solutions of subproblems that are boundaries of used solutions
\item Could be improved by solving bigger subproblems with less discarded ``boundary-problems'' in between
\item Solving subproblems of $k$ lines each only every $\frac{1}{k}$-th subproblem has to be discarded $\rightarrow$ $\left(1+\frac{1}{k}\right)$-- approximation.
\end{itemize}
\end{frame}
\end{document}



