% !TeX root = ../report.tex

\DeclareRobustCommand{\circN}[1]{\tikz[anchor=base,baseline=-.125cm]{\node(n){};\fill (n) circle(0.15);}}
\DeclareRobustCommand{\circI}[1]{\tikz[anchor=base,baseline=-.125cm]{\node(n){};\fill (n) circle(0.15);\draw[ultra thick, MidnightBlue](n)circle(0.25);}}
\DeclareRobustCommand{\circX}[1]{\tikz[anchor=base,baseline=-.125cm]{\node(n){};\fill[gray] (n) circle(0.1);}}

\begin{figure}[t]
\centering
\newcommand{\vv}{v_0}
    \begin{tikzpicture}[scale=0.8]
    \
    \coordinate (V) at (2,2);
    \coordinate (V2) at (3.6,2.2);
   \newcommand{\None}{(2,0.6),(0,2.1),(V2)}
   \newcommand{\Ntwo}{(5,1.5),(4,3.8)}

   {  
   	\fill (V) circle(0.1);
	\foreach \coord in \None{\fill \coord circle(0.1);}
	\foreach \coord in \Ntwo{\fill \coord circle(0.1);}
	\foreach \coord in \None{\draw (V) --\coord;}
	\foreach \coord in \Ntwo{\draw (V2) --\coord;}
	
}        
   {
	 \draw (V) circle(1);
   	 \foreach \coord in \None{\draw \coord circle(1);}
	\foreach \coord in \Ntwo{\draw \coord circle(1);}
}        
{\draw[] ($(V)+(-0.05,0.1)$)--+(-0.6,1)node[above]{$\vv$};}       
\coordinate(CAP) at (3,1);
\node [below=1.25cm, align=flush center,text width=5cm] at (CAP){UD arrangement \& UDG};
\draw[white] (-1.4,5) rectangle(7.2,-2.5);\end{tikzpicture}
%%%%%
\begin{tikzpicture}[scale=0.8]
    \coordinate (V) at (2,2);
    \coordinate (V2) at (3.6,2.2);
   \newcommand{\None}{(2,0.6),(0,2.1),(V2)}
   \newcommand{\Ntwo}{(5,1.5),(4,3.8)}

      
   {

	\foreach \coord in \None{\fill[gray!50] \coord circle(0.1);}
	\foreach \coord in \Ntwo{\fill[gray!50] \coord circle(0.1);}
	\foreach \coord in \None{\draw[gray!50] (V) --\coord;}
	\foreach \coord in \Ntwo{\draw[gray!50] (V2) --\coord;}
   	 \foreach \coord in \None{\draw[gray!50] \coord circle(1);}
	\foreach \coord in \Ntwo{\draw[gray!50] \coord circle(1);}
     	\fill[] (V) circle(0.15);
     	\draw[thick] (V) circle(1);	
}   

   {  


\draw[ultra thick,MidnightBlue] (V) circle(0.25);
}  

\coordinate(CAP) at (3,1);
 \node [below=1.25cm, align=flush center,text width=5cm] at (CAP){$N_0$ and $I_0$};
\draw[white] (-1.4,5) rectangle(7.2,-2.5);\end{tikzpicture}
\\
%%%
%%%%%
    \begin{tikzpicture}[scale=0.8]
    \
\coordinate (V) at (2,2);
\coordinate (V2) at (3.6,2.2);
\newcommand{\None}{(2,0.6),(0,2.1),(V2)}
\newcommand{\Ntwo}{(5,1.5),(4,3.8)}
%\draw[thick] (V) circle(1);
\draw[thick,gray!50] (V) circle(1);       
\foreach \coord in \Ntwo{\draw[gray] \coord circle(1);}
\foreach \coord in \None{\draw[thick] \coord circle(1);}
\foreach \coord in \Ntwo{\draw[gray] (V2) --\coord;}
\foreach \coord in \None{\draw (V) --\coord;}
\fill (V) circle(0.15);
    
\foreach \coord in \Ntwo{\draw[thick] (V2) --\coord;}
\foreach \coord in \None{\draw[thick] (V) --\coord;}
\foreach \coord in \None{\draw[ultra thick,MidnightBlue] \coord circle(0.25);}
\foreach \coord in \None{\fill[] \coord circle(0.15);}
\foreach \coord in \Ntwo{\fill[gray!50] \coord circle(0.1);}

\coordinate(CAP) at (3,1);
\node [below=1.25cm, align=flush center,text width=5cm] at (CAP){$N_1$ and $I_1$}; 
\draw[white] (-1.4,5) rectangle(7.2,-2.5);\end{tikzpicture}
\begin{tikzpicture}[scale=0.8] % i guess this is the bist picture of them all. use this for copypasting later
\coordinate (V) at (2,2);
\coordinate (V2) at (3.6,2.2);
\newcommand{\None}{(2,0.6),(0,2.1),(V2)}
\newcommand{\Nonee}{(2,0.6),(0,2.1)}
\newcommand{\Ntwo}{(5,1.5),(4,3.8)}

%\draw[thick] (V) circle(1);
%\draw[thick] (V2) circle(1);
\foreach \coord in \Ntwo{\fill \coord circle(0.15);}
\foreach \coord in \None{\fill \coord circle(0.15);}
\draw[thick,gray!50] (V) circle(1);
\draw[thick,gray!50] (V2) circle(1);
\foreach \coord in \Nonee{\draw[thick] \coord circle(1);}
\foreach \coord in \Ntwo{\draw[thick] \coord circle(1);}
\foreach \coord in \Ntwo{\draw[thick] (V2) --\coord;}
\foreach \coord in \None{\draw[thick] (V) --\coord;}
\foreach \coord in \Nonee{\draw[ultra thick,MidnightBlue] \coord circle(0.25);}
\foreach \coord in \Ntwo{\draw[ultra thick,MidnightBlue] \coord circle(0.25);}
\fill (V) circle(0.15);
\fill (V2) circle(0.15);
%\fill[gray] (V) circle(0.15);
%\fill[gray] (V2) circle(0.15);
\coordinate(CAP) at (3,1);
\node[below=1.25cm, align=flush center,text width=5cm] at (CAP){$N_2$ and $I_2$};
\draw[white] (-1.4,5) rectangle(7.2,-2.5);
\end{tikzpicture}
\caption{Unit disk graph (with disk representation) and the sets calculated by the algorithm. Nodes in a neighborhood set $N_r$ are marked with \circN\ \ and nodes in independent sets $I_r$ are marked with \circI \ . Not yet considered nodes are marked with \circX\ \ .  The algorithm starts at $\vv$ and takes neighborhoods $N_r$ of increasing size $r$ until the size of the independent set $I_r\subset N_r$ is below a certain threshold.}\label{fig:udgalg}
\end{figure}
