% !TeX root = ../../Nucleation in the Ising Model.tex
    \begin{tikzpicture}[thick,>=latex,scale=1,l/.style={latex-latex},F/.style={-latex,very thick},v/.style={-latex',thick},
			vc/.style={-latex',thick,color=gray},d/.style={thin},w/.style={fill=white}] 
  \begin{scope}

 \foreach \i in {0, 0.5, ..., 2.5} {
	\draw(0,\i)--+(2.5,0);
	\draw(\i,0)--+(0,2.5);

	 \foreach \j in {0, 0.5, ..., 2.5} {
        	\visible<3->{	 \filldraw (\i,\j) circle (1pt);}
        	}
    }
 	\visible<4->{
	\draw(0,0)node[below]{0};
 	\draw(0.5,0)node[below]{1};
 	\draw(1,0)node[below]{2};
	\draw(1.5,2.5)node[above]{N-3};
	\draw(2.5,2.5)node[above]{N-1};
	\draw(2,-0.25)node[below]{\ldots};
	\visible<5-6>{\draw[red](1,1)--+(0.5,0);
			\draw[red](1,1)--+(-0.5,0);
			\draw[red](1,1)--+(0,0.5);
			\draw[red](1,1)--+(0,-0.5);
	}
        	\visible<5-6>{\filldraw[red](1,1) circle (2pt);
	}
        	\visible<6>{\filldraw[red](0.5,1) circle (1pt);
			\filldraw[red](1.5,1) circle (1pt);
			\filldraw[red](1,1.5) circle (1pt);
			\filldraw[red](1,0.5) circle (1pt);
	}

        	\visible<7->{\filldraw[red](0,0) circle (2pt);
	}
        	\visible<8->{\draw[red](0,0)--+(0.5,0);\draw[red](0,0)--+(0,0.5);
			\filldraw[red](0,2.5) circle (1pt);
			\filldraw[red](0.5,0) circle (1pt);
			\filldraw[red](0,0.5) circle (1pt);
			\filldraw[red](2.5,0) circle (1pt);
	}


\begin{scope}[scale=1]
\coordinate(O) at(5,0,0);
\only<9->{

\foreach \x in {0, 1, ..., 2} {
\foreach \y in {0, 1, ..., 2} {
\draw($(O)+(0,\x,-\y)$)--+(2,0,0);
\draw($(O)+(\x,0,-\y)$)--+(0,2,0);
\draw($(O)+(\x,\y,0)$)--+(0,0,-2);
	 \foreach \z in {-1, -2, ..., -2} {
        		\filldraw($(O)+ (\x,\y,\z)$) circle (1pt);
	}
	 \filldraw($(O)+ (\x,\y,0)$) circle (1pt);
}}}
\only<10->{
\filldraw[red]($(O)+(1,1,-1)$) circle (2pt);

\filldraw[red]($(O)+(0,1,-1)$) circle (1pt);
\filldraw[red]($(O)+(1,0,-1)$) circle (1pt);
\filldraw[red]($(O)+(2,1,-1)$) circle (1pt);
\filldraw[red]($(O)+(1,2,-1)$) circle (1pt);
\filldraw[red]($(O)+(1,1,0)$) circle (1pt);
\filldraw[red]($(O)+(1,1,-2)$) circle (1pt);

\coordinate(O) at(6,1,-1);
\draw[red](O)--+(0,1,0) ;
\draw[red](O)--+(0,0,-1) ;
\draw[red](O)--+(0,0,1) ;
\draw[red](O)--+(1,0,0) ;
\draw[red](O)--+(-1,0,0) ;
\draw[red](O)--+(0,-1,0) ;		
}
\end{scope}

}
    \end{scope}
    \end{tikzpicture}