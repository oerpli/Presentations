% !TeX root = ../Plots.tex
% ringSizeProgressionNT
\begin{tikzpicture}[scale=1.5]
\def\XDOMMAX{13}
\def\XDOMMIN{-2}
\begin{axis}[
	, xlabel={Ringsize}
	, ylabel={Months}
	, zlabel={Density}
	, xmin=0
	, xmax=11
	, xtick={1,...,10},
	, zmin=0
	, legend style={nodes={scale=0.65, transform shape}}
	, area plot/.style={
        fill opacity=0.65,
        waterFall=#1,
        mark=none,
    }
]
\pgfplotstableread{./data/ringsizes/ringsizesNontrivial.csv}\ringsizesNT
% \pgfplotstablegetrowsof\ringsizes
% \pgfmathsetmacro\numberofrows{\pgfplotsretval-1}

\addplot3 [black, thick] table [x = avg, y = age, z expr = 0]{\ringsizesNT};
\addlegendentry{Average ringsize}
\addplot3 [red, thick] table [x = effavg, y = age, z expr = 0]{\ringsizesNT};
\addlegendentry{Effective avg ringsize}
\pgfplotsinvokeforeach{1,...,10}{
	\addplot3 [area plot=#1] table [x expr = #1, y=age, z=ring#1]{\ringsizesNT};
 }
% \addplot3 [area plot=#1] (x,#1,{normal(#1,2-0.1*#1)}) --(axis cs:\XDOMMAX,#1,0)--(axis cs:\XDOMMIN,#1,0);
% \pgfplotsinvokeforeach{4,3,...,1}{
%     \addplot3 [area plot=#1] table [x expr=\coordindex, y expr=#1, z=plot#1] {\dummydata};
% }
\end{axis}
\end{tikzpicture}